\documentclass[11pt]{tingpset}

\Name                   {Michaels Tingley and Traver}
\Email                  {\{michaeltingley, mtraver\}@college.harvard.edu}
\Organization           {Harvard University}
\Class                  {Computer Science 175}
\ClassShort             {CS175}
\ProblemSetNumber       {9}
\DueDate                {25 November 2013, 11:59pm}
\CollaborationStatement {Michaels Tingley and Traver completed this problem set in tandem.}

\begin{document}
  \makeheader

  %%%%%%%%%%%%%%%%%%%%%%%%%%%% 1 %%%%%%%%%%%%%%%%%%%%%%%%%%%%
  \problem
    Image \textbf{B} is the only viable image. Here's why.

    First, let's consider what it means to change the field of view. As read in section 10.3, a change in the field of view of $\theta$ degrees is the same as applying a scale of $-n = \frac{1}{\tan \frac{\theta}{2}}$. Thus, any transformation of the view of the cube must be consistent with a scale. Thus, it is clear that \textbf{B} is consistent with this --- as the front face of the cube grows in size, the rear face grows slower. However, the growth speed of both is proportional. That is, if the front face grows by 20\%, the back also appears to grow by 20\%. This is consistent with a change of scale.

    Meanwhile, this proportional growth rate is not maintained by answers \textbf{A} and \textbf{C}. In answer \textbf{A}, the rear face does not grow at all. In addition to this not being consistent, it begs the following question: \emph{what is happening on behind the rear face?} That is, if we had a third face that was behind the rear cube face, it's unclear what it would do... answer \textbf{A} may even imply that it would \emph{shrink}, which is clearly wrong. Answer \textbf{C} also doesn't make sense, because the back face grows faster than the front face. This doesn't make any sense; in addition to this not being consistent with proportionate growth, it begs the following question: \emph{what happens if we decreased the field of view even more?} If it were to continue with this trend, the rear face would grow larger the the front face... which doesn't make geometric sense at all.

  %%%%%%%%%%%%%%%%%%%%%%%%% 2, 3, 4 %%%%%%%%%%%%%%%%%%%%%%%%%
  \problemn{2, 3, \& 4}
    For these problems, we used the simplified projection matrix specified in the assignment, since it was simpler to work with and would lead to the same result.

  %%%%%%%%%%%%%%%%%%%%%%%%%%%% 2 %%%%%%%%%%%%%%%%%%%%%%%%%%%%
  \problem
    \ea{
      PS &=& \begin{pmatrix}
        1 & 0 & 0 & 0 \\
        0 & 1 & 0 & 0 \\
        0 & 0 & 0 & 1 \\
        0 & 0 & -1 & 0
      \end{pmatrix}
      \times
      \begin{pmatrix}
        3 & 0 & 0 & 0 \\
        0 & 3 & 0 & 0 \\
        0 & 0 & 3 & 0 \\
        0 & 0 & 0 & 3
      \end{pmatrix} \n
      \begin{pmatrix}
        1 & 0 & 0 & 0 \\
        0 & 1 & 0 & 0 \\
        0 & 0 & 0 & 1 \\
        0 & 0 & -1 & 0
      \end{pmatrix}
      \times
      3\mathbb{I} \n
      3P
    }

    The $S$ matrix above is just three times the identity matrix, so we're simply scaling $P$ by a constant. However, this has no effect on the scene. Why?

    Consider what happens when we scale the projection matrix. The $x$ and $y$ dimensions of everything in the scene get scaled, making them appear bigger to our eye. However, since we are also scaling the $z$ dimension, this has the effect of proportionally ``pushing things back''. So even though they get bigger, they get proportionally farther away, and so they look the same. Since we're also scaling the near and far plane, the clipping distances get scaled as well, and so there is no change with where objects get clipped, either.

  %%%%%%%%%%%%%%%%%%%%%%%%%%%% 3 %%%%%%%%%%%%%%%%%%%%%%%%%%%%
  \problem
    \ea{
      PQ &=& \begin{pmatrix}
        1 & 0 & 0 & 0 \\
        0 & 1 & 0 & 0 \\
        0 & 0 & 0 & 1 \\
        0 & 0 & -1 & 0
      \end{pmatrix}
      \times
      \begin{pmatrix}
        3 & 0 & 0 & 0 \\
        0 & 3 & 0 & 0 \\
        0 & 0 & 3 & 0 \\
        0 & 0 & 0 & 1
      \end{pmatrix} \n
      \begin{pmatrix}
        3 & 0 & 0 & 0 \\
        0 & 3 & 0 & 0 \\
        0 & 0 & 0 & 1 \\
        0 & 0 & -3 & 0
      \end{pmatrix} \n
      3 \begin{pmatrix}
        1 & 0 & 0 & 0 \\
        0 & 1 & 0 & 0 \\
        0 & 0 & 0 & \frac{1}{3} \\
        0 & 0 & -1 & 0
      \end{pmatrix}
    }
    Before we go through explanation of this, let's first recall the generalized depth-sensitive model for transforming to eye-coordinates. This is given by:
    \ea{
      \begin{pmatrix}
        x_c \\
        y_c \\
        z_c \\
        w_c
      \end{pmatrix}
      &=&
      \begin{pmatrix}
        s_x & 0 & -c_x & 0 \\
        0 & s_y & -c_y & 0 \\
        0 & 0 & \alpha & \beta \\
        0 & 0 & -1 & 0
      \end{pmatrix}
      \begin{pmatrix}
        x_e \\
        y_e \\
        z_e \\
        1
      \end{pmatrix}
    }

    As noted in the previous question, scaling the entire projection matrix by a constant has no effect on the project itself --- it is only the \emph{ratios} that matter. Thus, we have made $\beta$ to be one third the relative size as it was before. Since we have chosen to let $\beta = -\frac{2fn}{f-n}$, we realize that scaling up the value of $\beta$ will scale the near and far plane closer, and scaling down $\beta$ will scale these planes farther away.

    Here, we are scaling $\beta$ down, which scale the near and far plane closer. This results in having a ``closer'' clipping distance: we will be able to see images that are close to the camera that would normally get clipped out, and images that are farther away will get clipped earlier.

  %%%%%%%%%%%%%%%%%%%%%%%%%%%% 4 %%%%%%%%%%%%%%%%%%%%%%%%%%%%
  \problem

  \todo

  %%%%%%%%%%%%%%%%%%%%%%%%%%%% 5 %%%%%%%%%%%%%%%%%%%%%%%%%%%%
  \problem

  \todo

  %%%%%%%%%%%%%%%%%%%%%%%%%%%% 6 %%%%%%%%%%%%%%%%%%%%%%%%%%%%
  \problem

  \todo

  %%%%%%%%%%%%%%%%%%%%%%%%%%%% 7 %%%%%%%%%%%%%%%%%%%%%%%%%%%%
  \problem

  \todo

\end{document}
