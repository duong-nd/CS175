\documentclass[11pt]{tingpset}

\Name                   {Michaels Tingley and Traver}
\Email                  {\{michaeltingley, mtraver\}@college.harvard.edu}
\Organization           {Harvard University}
\Class                  {Computer Science 175}
\ClassShort             {CS175}
\ProblemSetNumber       {1.5}
\DueDate                {30 September 2013, 11:59pm}
\CollaborationStatement {Michaels Tingley and Traver completed this problem set in tandem.}

\begin{document}
  \makeheader

  %%%%%%%%%%%%%%%%%%%%%%%%%%%% 4.1 %%%%%%%%%%%%%%%%%%%%%%%%%%%%
  \problemn{4.1}

  %%%%%%%%%%%%%%%%%%%%%%%%%%%% 4.2 %%%%%%%%%%%%%%%%%%%%%%%%%%%%
  \problemn{4.2}

  %%%%%%%%%%%%%%%%%%%%%%%%%%%% 4.3 %%%%%%%%%%%%%%%%%%%%%%%%%%%%
  \problemn{4.3}
    \renewcommand{\a}{\vec{\boldsymbol{a}}^t}
    \renewcommand{\b}{\vec{\boldsymbol{b}}^t}
    Note that in both of these scenarios, we have to rotate by $\theta$. So we have no choice but to set
    \[
      R=
        \begin{pmatrix}
          \cos{\theta} & -\sin{\theta} & 0 \\
          \sin{\theta} & \cos{\theta} & 0 \\
          0 & 0 & 1
       \end{pmatrix}
    \]

    Now, we'll discuss $T$ in the equation $\b=\a TR$.

    We'll discuss this in two ways. First, the local way gives the following interpretation:



  %%%%%%%%%%%%%%%%%%%%%%%%%%%% 4.4 %%%%%%%%%%%%%%%%%%%%%%%%%%%%
  \problemn{4.4}

\end{document}
